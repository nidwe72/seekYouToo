%% LyX 2.3.2 created this file.  For more info, see http://www.lyx.org/.
%% Do not edit unless you really know what you are doing.
\documentclass[english]{article}
\usepackage[T1]{fontenc}
\usepackage[latin9]{inputenc}
\usepackage{mathrsfs}

\makeatletter
%%%%%%%%%%%%%%%%%%%%%%%%%%%%%% User specified LaTeX commands.


\usepackage{physics}
\usepackage{latexsym} 
\usepackage{ mathrsfs }

\makeatother

\usepackage{babel}
\begin{document}
\title{Quantum chromodynamics on your smartphone using 'SeekYouToo' : A workman
guide}

\maketitle
\textbf{Abstract}. Monte Carlo studies of non-ablian gauge theories
emerged in the early eighties now establishing a standard tool of
modern lattice quantum field theory. Due to the dramatic increase
of mobile computational power in the 2010s it is now possible to carry
out very basic simulations on smart phones. The author describes an
according cross-platform implementation of $SU_{2}$ gluon dynamics
and discusses some ditactical aspects.

\section*{A basic scetch of quantum chromodynamics}

\subsection*{Electrodynamics by concepts}

Quantum chromodynamics (QCD) describes the strong interaction of quarks
and gluons. Quarks make up hardons such as the protons and neutron
forming themeselves the nuclei of atoms. Gluons glue together the
quarks implementing the strong force. Quarks have spin 1/2 and thus
are fermions (i.e. particles with half odd integer spin) in constrast
to gluon carrying spin 1 and being bosons (particles with integer
spin). Spin plays an esential statistical role : Fermions obey the
Pauli exclusion principle stating that two identical fermions cannot
occupy the same quantum state simultaneously. Thus electrons live
in distinct shells around the nuclei of atoms. Gluons play the same
role as photons, being the carriers of electromagnetic force, yet
gluons can also interact with themselves.

Electrodynamics describes the electromagnetics fields. Historically
Gauss, Ampere and Faraday formulated distinct equations for the electric
and magntics fields, $\mathbf{E}$ and $\mathbf{B}$.

The magnetic field has no monopoles and thus divergenceless as expressed
in Gauss's law $\divergence{\vb{B}}=0$ of the magntic field. As the
divergence of a curl always vanishes the magnetic field can be written
as the curl of its potential $\vb{B}=\curl{\vb{A}}$.

Faradays law states a changing magnetic fileld induces a voltage in
a coil. That is the path integral of the electric field around a closed
loop equal to the negative rate if the change of the magnetic flux
through the area enclosed by the loop: $\oint_{C}\vb{E}\cdot d\vb{r}=-\partial\vb{B}/\partial t$.
Put in vector notation we have $\curl{\vb{E}}=-\partial\vb{B}/\partial t$.

Using the magnetic potential $\vb{A}$ we have $\curl{(\vb{E}+\partial\vb{A}/\partial t)}=0$.
When the curl of a vector field vanishes the field can always be written
as the gradient of a scalar, giving $\vb{E}=-\grad{V}-\partial\vb{A}/\partial t$
with $V$ being the electric potential.

As the electric and magnetic field only depends on the changes of
the accordings potentials we assume it is possible that we can change
the potentials and obtains the same fields. If we make the transformation
$\vb{A}\rightarrow\vb{A}+\grad{\lambda}$ the magnetic field $\vb{B}$
does not change since the curl of a gradient vanishes, $\vb{B}=\curl{\vb{A+\grad{\lambda}}}=\curl{\vb{A}}$.
However the electric field is changed, $\vb{E}=-\grad{V}-\partial\vb{A}/\partial t-\grad{\partial\lambda/\partial t}$
$=-\grad{(V+\partial\lambda/\partial t)}-\partial\vb{A}/\partial t$.\\
If another change $V\rightarrow V-\partial\lambda/\partial t$ is
made also $\vb{E}$ remains unchanged. A partial choice of the potentials
is called a gauge and can be done in many ways.

Gauss's law for the electric field $\divergence{\vb{E}}=\rho/\epsilon_{0},$and
Ampere's law $\curl{\vb{B}}=\mu_{0}\vb{J}+\mu_{0}\epsilon_{0}\partial\vb{E}/\partial t$
have not been mentioned yet completing the set of the of the ``four
macroscopic Maxwell equations''. Using an appropriate gauge condition,
the Lorenz (Ludvig) gauge $\divergence{A}+\mu_{0}\epsilon_{0}(\partial V/\partial t)=0$
lets one write the inhomogeneous Maxwell equations in a consise fashion
\begin{eqnarray*}
\Box^{2}V & = & \mu_{0}\epsilon_{0}\frac{\partial V}{\partial t}-\nabla{{}^2}V=\frac{\rho}{\epsilon_{0}}\\
\Box^{2}\vb{A} & = & \mu_{0}\epsilon_{0}\frac{\partial\vb{A}}{\partial t}-\nabla^{2}\vb{A}=\mu_{0}\vb{J}
\end{eqnarray*}
\\
Though the electric potential relates to static charges and the magnetic
one to moving charges the expressional symmetry of this appealing
formulation suggests that the distinction between electric and magnetic
fields are a question about point of view and that the electric and
magnetic fields form one field, the electromagnetic field.

Maxwell carried out a relativistic formulation taking into account
Lorentz (Hendrik) transformations. That is, he analysed how the fields
transform when changing from a coordinate frame in Minkoskian spacetime
to some other frame beeing boosted at constant velocity relative to
the original frame. These transformation can be represented by 4x4
real matrices $\Lambda$ (parametrized by the velocity $v$) such
that a transformed spacetime-vector has the same scalar (using the
Minkowski metric) as the original vector. The transfomation of the
electric and magnetic field can be written in a concise form letting
the Lorentz group matrices $\Lambda$ act on the electromagnetic field
tensor
\[
F_{\mu\nu}=\left[\begin{array}{cccc}
0 & -E_{x}/c & -E_{y}/c & -E_{z}/c\\
E_{x}/c & 0 & -B_{z} & B_{y}\\
E_{y}/c & B_{z} & 0 & -B_{x}\\
E_{z}/c & -B_{y} & B_{x} & 0
\end{array}\right]
\]

Using the four-vector potential given by $A_{\mu}=(V/c,A_{x},A_{y},A_{z})$
and the four-current defined as $J^{\nu}=(c\rho,J_{x},J_{y},J_{z})$
the inhomogeneous Maxwell equation can written in the very compact
form
\[
\partial_{\mu}F^{\mu\nu}=\mu_{0}J^{\nu}
\]

with

\[
F^{\mu\nu}=\partial^{\mu}A^{\nu}-\partial^{\nu}A^{\mu}
\]

Note that in the 4-notaion greek indices run from 0 to 3 and the Einstein
summation convention is applied when an index is repeated both lowered
and raised, that is for two 4-vectors $X^{\mu}$and $Y_{\mu}$ the
expression $X^{\mu}Y_{\mu}$ means $X^{\mu}Y_{\mu}=\sum_{\mu=0}^{3}X^{\mu}Y_{\mu}.$

Also note that a second order tensor (a matrix) $X_{\mu\nu}$ with
only lowered indices is called covariant and that according contravariant
tensor (raised indices) is build by applying the metric tensor $\eta$
\[
X^{\mu\nu}=\eta^{\mu\alpha}\eta^{\nu\beta}X_{\alpha\beta};\quad\eta^{\mu\nu}=\left[\begin{array}{cccc}
-1 & 0 & 0 & 0\\
0 & 1 & 0 & 0\\
0 & 0 & 1 & 0\\
0 & 0 & 0 & 1
\end{array}\right]
\]

One important feature of the 4-vector notation to be mentioned is
that the product of a contravariant and a covariant tensor $X^{\mu}X_{\mu}$
is invariant under Lorentz transformation.

It is easy to show that the Maxwell equation is the Euler-Lagrange
equation of the Lagrangian 
\[
{L}=\mathscr{{L}}_{field}+\mathscr{{L}}_{interaction}=-\frac{1}{4\mu_{0}}F_{\mu\nu}F^{\mu\nu}-J^{\mu}A_{\mu}.
\]


\subsection*{Gauge theories and symmetries}

Noether's theorem states that for every differentiable global symmetry
of the action of a Lagrangian system there is a corresponding conservation
law. For instance time translations give conversation of energy and
spatial translations yield conservation of enery. If we have for example
a a complex-valued scalar field $\phi$ being invariant under some
$U(1)$ transformation, $\phi(x)\rightarrow e^{i\alpha}\phi(x)$ where
$\alpha$ does not depend on position $x$ one can construct an according
conserved Noether current. If however we have a position-dependent
$\alpha(x)$ and the Lagrangain involves derivatives we obtain unwanted
term destroying the invariance, because at each point we have different
transformation roles. In order to have obtain local symmetry we tweak
the partial derivative introducing a new gauge potential $A_{\mu}$:
The covariant devirative 
\[
D_{\mu}=\partial_{\mu}-igA_{\mu}
\]
\\
eliminates the terms indroduced by the partial derivative, argueing
as follows. We assume that$A_{\mu}$ transforms as $A_{\mu}\rightarrow A\mu+(1/g)\partial\mu\alpha(x)$.
By comparing to the the gauge discussed for the electromegnetic field,
$V\rightarrow V-\partial\lambda/\partial t$ and $\vb{A}\rightarrow\vb{A}+\grad{\lambda}$)
, we see that $\lambda$ is equivalent to $\alpha(x)/g$ with $g$
being the coupling constant (being $e$ in the case of electrodynamics).
Using some algebra we see that the covariant derivative applied to
the phase transformed field together with the gauge transformation
gives us an expression in which the covariant derivative itself remains
does not depend on the phase:
\begin{eqnarray*}
D_{\mu} & \rightarrow & \left[\partial_{\mu}+ig\left(A_{\mu}-\frac{1}{g}\partial_{\mu}\alpha(x)\right)\right]e^{i\alpha(x)}\phi=\\
 & = & e^{i\alpha(x)}\partial_{\mu}\phi\\
 &  & +ie^{i\alpha(x)}\phi\partial_{\mu}\alpha(x)+igA_{\mu}e^{i\alpha(x)}\phi\\
 &  & -ie^{i\alpha(x)}\phi\partial_{\mu}\alpha(x)\\
 & = & e^{i\alpha(x)}\left(\partial_{\mu}-igA_{\mu}\right)\phi\\
 & = & e^{i\alpha(x)}D_{\mu}\phi
\end{eqnarray*}

As the derivative $D_{\mu}$ transforms the same way as the field
$\phi$ it is the name ``covariant'' suits pretty well.

Using covariant derivatives we can thus construct meaningful Lagrangians
preserving local symmetry. The simple real gauge theory is quantum
electrodynamics (QED) augmenting electrodynamics by fermions (for
instance electrons) described by Dirac spinors $\psi$:
\begin{eqnarray*}
{L} & = & {L}_{Dirac}+{L}_{Maxwell}+{L}_{interaction}\\
 & = & \bar{\psi}\left(i\gamma^{\mu}\partial\mu-m\right)\psi-\frac{1}{4}F_{\mu\nu}F^{\mu\nu}-e\bar{\psi}\gamma^{\mu}\psi A_{\mu}
\end{eqnarray*}
\\
Using the covariant derivative the Lagrangian reads
\[
{L}=\bar{\psi}\left(i\gamma^{\mu}D_{\mu}-m\right)\psi-\frac{1}{4}F_{\mu\nu}F^{\mu\nu}
\]
and one obtains as the equation of motions for the fermionic wave
function $\psi$ and the electromagnetic potential $A_{\mu}$
\begin{eqnarray*}
\left(i\gamma^{\mu}D_{\mu}-m\right)\psi & = & 0\\
\partial_{\mu}F^{\mu\nu} & = & e\bar{\psi}\gamma^{\nu}\psi
\end{eqnarray*}

The symmetry group of QED is $U(1)$, that is the Lagrangian is invariant
under $\psi\rightarrow e^{i\alpha(x)}\psi$ and the according gauge
field is the electromagnetic potential.

$U(1)$ is a simple non-trivial symetry group and we now turn to the
case to more involved case that the groups in question are non-abelian.
Yang and Mills therefore considered transformations of the form $\phi\rightarrow\exp\left(i\alpha^{a}t^{a}\right)\phi$
where ...

\subsection*{Quantum chromodynamics}
\end{document}
